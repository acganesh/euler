\documentclass{article}

\title{Problem 93 - Arithmetic Expressions}
\begin{document}
\maketitle
Note that for the set $\{1, 2, 3, 4\}$, the integers from $1$ to $28$ can be created using combinations of the basic arithmetic expressions and brackets.  Our goal is to determine the the set of four digits $\{a, b, c, d\}$ such that the integers from $1$ to $N$ can be created, where $N$ is maximal over all possible sets.

This seems straightforward to brute force, but let's first look at a few simple cases to see how this function behaves.

\[
3*1 - 4/2  = 1
\]
\[
2*3 - 4*1 = 2
\]
\[
(4+2)/(3-1) = 3
\]

It doesn't seem like there's an obvious way to compute $N$ analytically, but we just need to iterate over all possible strings, and then compute our desired value. Use postfix!

\end{document}
