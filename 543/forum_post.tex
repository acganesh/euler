
I thought I'd revisit this problem and implement Lehmer's algorithm to compute $\pi(x)$.  My previous algorithm using a sieve to count primes ran in ~40 seconds, but applying Lehmer's algorithm shortened it to 0.3 seconds!

Here's an overview of the algorithm.  Let $p_1, p_2, \dots, p_a$ denote the first $n$ primes, and let $\Phi(x, a)$ denote the number of natural numbers not greater than $m$ which are divisible by no $p_i$.  It follows that 

$\Phi(x, a) = \Phi(x, a-1) + \Phi \left ( \frac{x}{p_{a}}, a-1 \right )$

Then $\pi(x)$ can be computed as
$ \pi(x) = \Phi(x, a) + \frac{1}{2} (b+a-2)(b-a+1) + \sum_{a < i \leq b} \pi \left ( \frac{x}{p_i} \right ) - \sum_{i=a+1}^{c} \sum_{j=1}^{b_i} \left [ \pi \left ( \frac{x}{p_i p_j} \right ) - (j-1) \right ]$
,

where:

$a = \pi(x^{(1/4)})$
$b = \pi(x^{(1/2)})$
$b_i = \pi(\sqrt{x/p_i})$
$c = \pi(x^{1/3})$

Here's my Python code implementing this, adapted from this [url=http://stackoverflow.com/a/19072704]Stack Overflow thread[/url].  A good reference that discusses this algorithm and other related ones is Prime Numbers and Computer Methods for Factorization by H. Riesel.  
